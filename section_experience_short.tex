% Awesome CV LaTeX Template
%
% This template has been downloaded from:
% https://github.com/huajh/huajh-awesome-latex-cv
%
% Author:
% Junhao Hua


%Section: Work Experience at the top
\sectionTitle{Projects \& Experiences}{\faCode}

\begin{experiences}


	\experience
	{Mar 2014}   {Custom Engine Control}{ITS}{C}
	{Sept 2014} {
	                  \begin{itemize}
	                    \item Kendalikan timing Busi dan volume injector
	                    \item Membaca TPS dan RPM
	                    \item PCB buatan sendiri berbasis chip STM32F103RE dan FET IRF540
	                    \item \faGithub: \link{https://github.com/mekatronik-achmadi/ecu_gea_v1}{github.com/mekatronik-achmadi/ecu\_gea\_v1}
	                  \end{itemize}
	                }
	                {KiCAD, STM32, C, RPM, TPS, Coil, Injector}
	\emptySeparator

	\experience
	{Jul 2019} {Custom Audiometry}{ITS}{C/Python}
	{Oct 2023}    {
		\begin{itemize}
			\item Implementasi Metode Audiometri 3-FC untuk portable device
			\item Berbasis STM32F401RE atau STM32F303RB dengan audio MAX98357A
			\item Fitur IoT dengan tambahan ESP32
			\item Development PCB produk sudah mencapai minimal TKT Level 7
			\item \faGithub: \link{https://github.com/VibrasticLab/pikoakustik} {github.com/VibrasticLab/pikoakustik}
		\end{itemize}
	}
	{KiCAD, STM32, C, Python, Audiometry, Medical Device, IoT}
	\emptySeparator

%	\experience
%	{Jun 2020} {Custom Cough Analyzer}{ITS}{C/Python}
%	{Sept 2020}    {
%		\begin{itemize}
%			\item Implementasi Pemrosesan Sinyal untuk Audio Batuk
%			\item Berbasis RaspberryPi 4 dengan microphone INMP441
%			\item Data audio batuk diolah di server yang running Tensorflow
%			\item \faGithub:
%			\link{https://github.com/VibrasticLab/ehealth-iot} {github.com/VibrasticLab/ehealth-iot/},
%			\link{https://github.com/VibrasticLab/ehealth-web} {github.com/VibrasticLab/ehealth-web}
%		\end{itemize}
%	}
%	{KiCAD, RaspberryPi, C, Python, Cough, Medical Device, IoT, Web AI}
%	\emptySeparator
	
	\experience
	{Apr 2023} {Custom Engine Control}{Poltek Madiun}{C/C++}
	{Nov 2023}    {
		\begin{itemize}
			\item Kendalikan timing Busi dan volume injector
			\item Membaca TPS dan RPM
			\item PCB buatan sendiri berbasis chip STM32F051RB dan FET IRF540
			\item \faGithub: \link{https://github.com/deninur2427/ecu_pnm}{github.com/deninur2427/ecu\_pnm}
		\end{itemize}
	}
	{KiCAD, STM32, C, C++, RPM, TPS, Coil, Injector}
	\emptySeparator
	
	\experience
	{Nov 2021} {Programming Training}{ITS}{C/Python}
	{Present}    {
		\begin{itemize}
			\item Menulis tutorial dalam format Markdown atau \LaTeX
			\item Melatih dasar kemampuan programming di C, C++, atau Python
			\item \faGithub: \link{https://github.com/mekatronik-achmadi/md_tutorial}{github.com/mekatronik-achmadi/md\_tutorial}
		\end{itemize}
	}
	{Markdown, Training}
	\emptySeparator

\end{experiences}
